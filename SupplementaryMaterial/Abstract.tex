% Place abstract below.

\vspace{-0.3in}

Neutrinos are fundamental particles in the standard model and play an important role in the current understanding of the universe; however, the mass of the neutrino one of the most fundamental parameters for any particle, is currently unknown. This fact represents a gaping hole in our current knowledge of the universe that may provide clues to the energy scale of physics beyond the standard model. This dissertation summarizes research and development as a member of the Project 8 collaboration towards an experiment to measure the neutrino mass with a sensitivity below $50$~$\mathrm{meV}/\mathrm{c}^2$, which is an order of magnitude less than the most sensitive direct measurements of the neutrino mass to date. Project 8 will perform this measurement using Cyclotron Radiation Emission Spectroscopy (CRES) to measure the beta-decay endpoint spectrum of atomic tritium. I present an analysis of the signal reconstruction performance of an antenna array system designed to perform large-scale CRES measurements in cubic-meter volumes. Next, I discuss an approach to calibrating an antenna-based CRES experiment using a unique probe antenna designed to mimic radiation from CRES events. Finally, I present design studies for a resonant cavity that could be used to perform a CRES experiment with atomic tritium at multi-cubic-meter scales.