% !TEX root = ../YourName-Dissertation.tex

\chapter{Development of Resonant Cavities for Large Volume CRES Measurements}

\section{Introduction}

The cavity approach is an alternative CRES measurement technology under consideration by the Project 8 collaboration for a neutrino mass measurement experiment with 40~meV sensitivity. After pursuing an antenna array based CRES demonstrator design for several years the increasing costs and complexity of the antenna arrays led to a rexamination of resonant cavities for a large scale experiment. Currently, a cavity based CRES experiment is the preferred technology for the Project 8 neutrino mass measurement goal with antennas as a fall-back approach. 

In this chapter we provide a brief summary of resonant cavities and sketch out the key features of a cavity based CRES experiment. In Section \ref{sec:chap6-resonant-cavities} we provide a brief introduction to cylindrical resonant cavities and the solutions for the electromagnetic fields in the cavity volume.

In Section \ref{sec:chap6-cavity-approach} we describe the main components of a cavity based CRES experiment. Including the background and trap magnets, cavity geometry and design, and cavity coupling considerations. We also discuss some relevant trade-offs between an antenna array and cavity CRES experiment, and highlight some reasons for the transition of Project 8 to the development of a cavity based experiment.

Finally, in Sections \ref{sec:chap6-single-mode-cavity-sims} and \ref{sec:chap6-single-mode-cavity-measurement} I present the design and development of an open, mode-filtered cavity that could be used in a cavity based CRES experiment. The results of the cavity simulations are confirmed by laboratory measurements of a proof-of-principle prototype cavity intended to demonstrate key features of the design.

\section{Cylindrical Resonant Cavities}
\label{sec:chap6-resonant-cavities}

Resonant cavities are essentially sealed conductive containers, which allows us to describe the electromagnetic (EM) fields as a superposition of resonant modes. The field shapes of the resonant modes are determined by Maxwell's equations and the boundary conditions enforced by the cavity geometry. Of interest to Project 8 for CRES measurements are cylindrical cavities due to their ease of construction and integration with atom and electron trapping magnets. 

\subsection{General Field Solutions}

Consider a long segment of conducting material with a cylindrical cross-section (see Figure \ref{fig:chap6-circ-waveguide}). A geometry such as this can be used as a waveguide transmission line to transfer EM energy from point to point, or, if conducting shorts are inserted on both ends of the cylinder, the waveguide becomes a resonant cavity. 

\begin{figure}[htbp]
    \centering
    \includegraphics*[width=0.4\textwidth]{figs/Chapter-6/230606_circular_waveguide.png}
    \caption{\label{fig:chap6-circ-waveguide} Geometry of a cylindrical waveguide with radius $b$. }
\end{figure}

The fields allowed inside a cylindrical cavity are determined by the boundary conditions of the cylindrical geometry. The general approach to solving for the fields begins by assuming solutions to Maxwell's equations of the form
\begin{align}
    \label{eq:chap6-maxwell-eqn-solutions-E}\bm{E}(x,y,z)&=(\bm{e}(x,y)+\hat{z}e_z(x,y))e^{-i\beta z},\\
    \label{eq:chap6-maxwell-eqn-solutions-H}\bm{H}(x,y,z)&=(\bm{h}(x,y)+\hat{z}h_z(x,y))e^{-i\beta z}.
\end{align}

The solutions assume a harmonic time dependence of the form $e^{i\omega t}$ and propagation along the positive z-axis. The functions $\bm{e}(x,y)$ and $\bm{h}(x,y)$ represent the transverse ($\hat{x}, \hat{y}$) components of the electric and magnetic fields respectively, and $e_z(x,y)$, $h_z(x,y)$ are the longitudinal components. The version of Maxwell's equations in the case where there are no source terms can be written as a pair of coupled differential equations, 
\begin{align}
    \nabla\times\bm{E}&=-i\omega\mu\bm{H},\\
    \nabla\times\bm{H}&=i\omega\epsilon\bm{E},
\end{align}
where $\epsilon$ and $\mu$ are the permittivity and permeability of the material inside the waveguide or cavity. Using the field solutions from Equations \ref{eq:chap6-maxwell-eqn-solutions-E} and \ref{eq:chap6-maxwell-eqn-solutions-H} one can solve for the transverse components of the fields in terms of the longitudinal fields. Because we are interested in cylindrical cavities it is advantageous to write the field solutions in cylindrical coordinates. After performing this transformation the set of four equations for the transverse field components are,
\begin{align}
    H_\rho&=\frac{i}{k_c^2}\left(\frac{\omega\epsilon}{\rho}\frac{\partial E_z}{\partial \phi}-\beta \frac{\partial H_z}{\partial \rho}\right),\\
    H_\phi&=\frac{-i}{k_c^2}\left(\omega\epsilon\frac{\partial E_z}{\partial \rho}+\frac{\beta}{\rho}\frac{\partial H_z}{\partial \phi}\right),\\
    E_\rho&=\frac{-i}{k_c^2}\left(\beta\frac{\partial E_z}{\partial \rho}+\frac{\omega\mu}{\rho}\frac{\partial H_z}{\partial \phi}\right),\\
    E_\phi&=\frac{i}{k_c^2}\left(\frac{-\beta}{\rho}\frac{\partial E_z}{\partial \phi}+\omega\mu\frac{\partial H_z}{\partial \rho}\right),
\end{align}
where $k_c$ is the cutoff wavenumber defined by $k_c^2=k^2-\beta^2$ with $k=\omega\sqrt{\mu\epsilon}$ being the wavenumber of the EM radiation. 

This set of equations can be used to solve for a variety of different modes that can be obtained by setting conditions on $E_z$ and $H_z$. For cylindrical cavities two types of modes are allowed, which correspond to solutions where $E_z=0$ and $H_z=0$ respectively. 

\subsection{TE and TM Modes}
\label{sec:chap6-TE-TM-modes}

The TE family of modes corresponds to the case where $E_z=0$. This implies that $H_z$ is a solution to the Helmholtz wave equation 
\begin{equation}
    (\nabla^2 + k^2)H_z = 0.
    \label{eq:chap6-helmholtz-magnetic}
\end{equation}
For solutions of the form $H_z(\rho,\phi,z)=h_z(\rho,\phi)e^{-i\beta z}$, Equation \ref{eq:chap6-helmholtz-magnetic} can be solved using the standard technique of separation of variables. Rather than reproduce the derivation here we shall simply quote the solutions for the transverse fields, which are 
\begin{align}
    H_\rho &= \frac{-i\beta }{k_{c_{nm}}}(A\sin{n\phi}+B\cos{n\phi})J_n^\prime(k_{c_{nm}}\rho)e^{-i\beta_{nm} z},\\
    H_\phi &=\frac{-i\beta n}{k_{c_{nm}}^2\rho}(A\cos{n\phi}-B\sin{n\phi})J_n(k_{c_{nm}}\rho)e^{-i\beta_{nm} z},\\
    E_\rho &=\frac{-i\omega\mu n}{k_{c_{nm}}^2 \rho}(A\cos{n\phi}-B\sin{n\phi})J_n(k_{c_{nm}}\rho)e^{-i\beta_{nm} z},\\
    E_\phi &=\frac{i\omega\mu}{k_{c_{nm}}}(A\sin{n\phi}+B\cos{n\phi})J_n^\prime(k_{c_{nm}}\rho)e^{-i\beta_{nm} z}.
\end{align}
One can observe that the solutions have a periodic dependence on $\phi$, and radial profiles given by the Bessel functions of the first kind. The integer indices $n$ and $m$ arise from continuity conditions on the EM fields in the azimuthal and radial directions. For the TE modes $n\geq0$ and $m\geq1$. $k_{c_{nm}}$ is the cutoff wavenumber for the $\mathrm{TE}_{nm}$ mode given by 
\begin{equation}
    k_{c_{nm}} = \frac{p^\prime_{nm}}{b},
\end{equation}
where $b$ is the radius of the cavity or waveguide and $p^\prime_{nm}$ is the $m$-th root of the derivative of the $n$-th order Bessel function (see Table \ref{tab:chap6-bessel-derivative roots}).

\begin{table}[htbp]
    \centering
    \caption{\label{tab:chap6-bessel-derivative-roots} A table of the values of $p_{nm}^\prime$.}
    \begin{tabular}{c c c c }
        \hline
        $n$ & $p_{n1}^\prime$ & $p_{n2}^\prime$ & $p_{n3}^\prime$ \\
        \hline
        0 & 3.832 & 7.016 & 10.174\\
        1 & 1.841 & 5.331 & 8.536\\
        2 & 3.054 & 6.706 & 9.970\\
        \hline
    \end{tabular}
\end{table}

The TM mode family corresponds to the case where $H_z=0$, and $(\nabla^2 +k^2)E_z=0$. Again, we assume solutions of the form $E_z(\rho,\phi,z)=e_z(\rho,\phi)e^{-i\beta z}$, for which the general form of the solutions is the same as for the TE modes. However, the different boundary conditions for the TM modes results in particular solutions with a different from, which we shall quote here without derivation. The transverse fields of the TM modes are given by 
\begin{align}
    H_\rho &=\frac{-i\omega\epsilon n}{k_{c_{nm}}^2 \rho}(A\cos{n\phi}-B\sin{n\phi})J_n(k_{c_{nm}}\rho)e^{-i\beta_{nm} z},\\
    H_\phi &=\frac{-i\omega\epsilon}{k_{c_{nm}}}(A\sin{n\phi}+B\cos{n\phi})J_n^\prime(k_{c_{nm}}\rho)e^{-i\beta_{nm} z}\\
    E_\rho &= \frac{-i\beta }{k_{c_{nm}}}(A\sin{n\phi}+B\cos{n\phi})J_n^\prime(k_{c_{nm}}\rho)e^{-i\beta_{nm} z},\\
    E_\phi &=\frac{-i\beta n}{k_{c_{nm}}^2\rho}(A\cos{n\phi}-B\sin{n\phi})J_n(k_{c_{nm}}\rho)e^{-i\beta_{nm} z},
\end{align}
which one may notice are the same solutions as the TE modes with $H$ and $E$ flipped. The cutoff wavenumber for the TM modes is given by, $k_{c_{nm}}=p_{nm}/b$, where the values of $p_{nm}$ correspond to the $m$-th zero of the $n$-th order Bessel function (see Table \ref{tab:chap6-bessel-roots}).

\begin{table}[htbp]
    \centering
    \caption{\label{tab:chap6-bessel-roots} A table of the values of $p_{nm}$.}
    \begin{tabular}{c c c c }
        \hline
        $n$ & $p_{n1}$ & $p_{n2}$ & $p_{n3}$ \\
        \hline
        0 & 2.405 & 5.520 & 8.654\\
        1 & 3.832 & 7.016 & 10.174\\
        2 & 5.135 & 8.417 & 11.620\\
        \hline
    \end{tabular}
\end{table}

\subsection{Resonant Frequencies of a Cylindrical Cavity}

A cylindrical cavity is essentially constructed by taking a section of cylindrical waveguide and shorting both ends with conductive material. This means that the electric fields inside a cylindrical cavity are exactly those we derived in Section \ref{sec:chap6-TE-TM-modes} with the additional condition that the electric fields must go to zero at $z=0$ and $z=L$ (see Figure \ref{fig:chap6-cyl-cav}).
\begin{figure}[htbp]
    \centering
    \includegraphics*[width=0.3\textwidth]{figs/Chapter-6/230606_cylindrical_cavity.png}
    \caption{\label{fig:chap6-cyl-cav} The geometry of a cylindrical cavity with length $L$ and radius $b$.}
\end{figure}

The transverse electric field solutions for a cylindrical waveguide are of the form
\begin{equation}
    \bm{E}(\rho,\phi,z)=\bm{e}(\rho,\phi)\left(A_+e^{-i\beta_{nm}z}+A_-e^{i\beta_{nm}z }\right),
\end{equation}
where $A_+$ and $A_-$ are arbitrary amplitudes of forward and backward propagating waves. In order to enforce that $\bm{E}$ is zero at both ends of the cavity we require that 
\begin{equation}
    \beta_{nm}L = 2\pi \ell,
\end{equation}
where $\ell=0,1,2,3\ldots$. Using this constraint on the propagation constant we can solve for the resonant frequencies of the $\mathrm{TE}_{nm\ell}$ and the $\mathrm{TM}_{nm\ell}$ modes in a cylindrical cavity. For the TE modes the resonant frequencies are
\begin{equation}
    f_{nm\ell}=\frac{c}{2\pi\sqrt{\mu_r\epsilon_r}}\sqrt{\left(\frac{p_{nm}^\prime}{b}\right)^2+\left(\frac{\ell\pi}{L}\right)^2},
\end{equation}
and the frequencies of the TM modes are 
\begin{equation}
    f_{nm\ell}=\frac{c}{2\pi\sqrt{\mu_r\epsilon_r}}\sqrt{\left(\frac{p_{nm}^\prime}{b}\right)^2+\left(\frac{\ell\pi}{L}\right)^2}.
\end{equation}

\begin{figure}[htbp]
    \centering
    \includegraphics[width=0.75\textwidth]{figs/Chapter-6/230610_mode_lines.png}
    \caption{\label{fig:chap6-cavity-mode-lines} Relation of mode frequency to cavity length for a cylindrical cavity with a radius of 18.32~cm.}
\end{figure}

\subsection{Cavity Q-factors}

\begin{figure}[htbp]
    \centering
    \includegraphics*[width=0.6\textwidth]{figs/Chapter-6/230607_rlc.png}
    \caption{\label{fig:chap6-series-rlc} A series RLC circuit.}
\end{figure}

The resonant behavior of cylindrical cavities can be modeled as a series RLC circuit (see figure \ref{fig:chap6-series-rlc}). The input impedance of the circuit can be obtained by applying Kirchhoff's laws to calculate the impedance of the equivalent circuit. For a series RLC circuit the input impedance is 
\begin{equation}
    Z_\mathrm{in}=\left(\frac{1}{R}+\frac{1}{i\omega L}+i\omega C\right).
\end{equation}
The resistance in the circuit represents all sources of loss in the cavity, which is primarily caused by the finite conductivity of the cavity walls, and the inductor and capacitor represent the energy stored in the cavity in the form of electric and magnetic fields. If the circuit is being driven by an external power source we can write the input power in terms of the circuit input impedance and the source voltage 
\begin{equation}
    P_\mathrm{in} = \frac{1}{2}Z_\mathrm{in}|I|^2=\frac{1}{2}|I|^2\left(\frac{1}{R}+\frac{1}{i\omega L}+i\omega C\right).
\end{equation} 
The resistor introduces a loss into the system with a power given by 
\begin{equation}
    P_\mathrm{loss} = \frac{1}{2}|I|^2R,
\end{equation}
and the capacitor and inductor store energies given by
\begin{align}
    W_e&=\frac{1}{4}\frac{|I|^2}{\omega^2 C},\\
    W_m&=\frac{1}{4}|I|^2 L,
\end{align}
respectively. Using these expressions we can write the input power and input impedance expressions in terms of the lost power and stored energy 
\begin{align}
    P_\mathrm{in}=P_\mathrm{loss}+2i\omega(W_m-W_e),\\
    Z_\mathrm{in}=\frac{P_\mathrm{loss}+2i\omega(W_m-W_e)}{\frac{1}{2}|I|^2}.
\end{align}

The condition for resonance in the RLC circuit is that the stored magnetic energy is equal to the stored electric energy ($W_e$=$W_m$). When this occurs $Z_\mathrm{in}=R$, which is a purely real impedance, and $P_{in}=P_{loss}$. The resonant frequency of the circuit can be determined from the condition $W_e=W_m$ from which one finds that 
\begin{equation}
    \omega_0=\frac{1}{\sqrt{LC}}.
\end{equation}
An important performance parameter for any resonant system is the Q-factor, which quantifies the quality of the resonator as the ratio of the stored energy multiplied by the resonant frequency to the average energy lost per second. For the series RLC circuit, the Q-factor is given by the expression 
\begin{equation}
    Q_0=\omega\frac{W_e+W_m}{P_\mathrm{loss}}=\frac{1}{\omega_0RC},
\end{equation}
from which one observes that as the resistance of the RLC circuit is decreased the quality factor of the resonator increases. From the perspective of cylindrical cavities this implies that as one decreases the resistance of the cavity walls it is expected that the Q-factor of the cavity should increase, which is indeed the case. In certain applications where a high Q is desireable it is possible to manufacture a cavity out of superconducting materials in order to minimize the power losses of the system.

\begin{figure}[htbp]
    \centering
    \includegraphics*[width=0.6\textwidth]{figs/Chapter-6/230607_rlc_resonance.png}
    \caption{\label{fig:chap6-rlc-resonance}}
\end{figure}

The Q-factor of the resonator also determines with bandwidth (BW) of the system. A cavity with a high Q-factor will resonant with a smaller range of frequencies than a cavity with a low Q-factor. To see this we can examine the behavior of the RLC circuit when driven by frequencies near the resonance. For a frequency $\omega=\omega_0+\Delta \omega$, where $\Delta \omega=\omega-\omega_0\ll \omega_0$, we can write the input impedance as 
\begin{equation}
    Z_\mathrm{in}=R+i\omega L\left(\frac{\omega^2-\omega_0^2}{\omega^2}\right),
\end{equation}
and by expanding $(\omega^2-\omega_0^2)/\omega^2$ to first order in $\Delta\omega$, we obtain 
\begin{equation}
    Z_\mathrm{in}\approx R+i\frac{2RQ_0\Delta\omega}{\omega_0}.
\end{equation}
Therefore, the magnitude of the input impedance near the resonance is given by 
\begin{equation}
    |Z_\mathrm{in}|=R\sqrt{1+4Q_0^2\frac{\Delta\omega^2}{\omega^2}}, 
    \label{eq:chap6-mag-input-impedance}
\end{equation}
from which we observe that for the series RLC circuit the input impedance is minimized at the resonant frequency, which corresponds to the maximum input power (see Figure \ref{fig:chap6-rlc-input-impedance}). The half-power BW is the range of frequencies over which the input power drops to half the input power on resonance. This occurs when $|Z_\mathrm{in}|=\sqrt{2}R$, which corresponds to $\Delta\omega/\omega=\textrm{BW}/2$. Using Equation \ref{eq:chap6-mag-input-impedance} one can find that
\begin{equation}
    2R^2=R^2(1+Q_0^2\mathrm{BW}^2),
\end{equation}
which implies
\begin{equation}
    \mathrm{BW}=\frac{1}{Q_0}
\end{equation}

\begin{figure}[htbp]
    \centering
    \includegraphics*[width=0.7\textwidth]{figs/Chapter-6/230607_loaded_rlc.png}
    \caption{\label{fig:chap6-loaded-resonator-circuit} A series RLC circuit coupled to an external circuit with input impedance $R_L$.}
\end{figure}

It is important to emphasize that the Q-factor defined here, $Q_0$, is technically the unloaded Q. It reflects the quality of the cavity or resonant circuit without the influence of any external circuitry. In practice, however, a cavity is invariably coupled to an external circuit to drive a cavity resonance or to measure the energy of a resonant mode. Coupling a cavity to an external circuit changes the Q by loading the equivalent cavity RLC circuit (see Figure \ref{fig:chap6-loaded-resonator-circuit}). The Q-factor of the cavity when it is loaded by an external circuit is called the loaded Q, which is the quantity that one actually measures when exciting a resonance in the cavity. Using the series RLC circuit model one can see that the load resistor in Figure \ref{fig:chap6-loaded-resonator-circuit} will add in series with the resistor in the circuit for a total equivalent resistance of $R+R_L$. Therefore, the loaded Q is given by 
\begin{equation}
    Q_L=\frac{1}{\omega_0(R+R_L)C},
\end{equation}
from which one observes that the loaded Q is always less than the intrinsic Q of the cavity.

The amount of coupling that is desireable depends on the specific application of the resonator. If one wants a resonator that is particular frequency selective than it makes sense to limit the amount of coupling to the cavity to maintain a small BW, alternatively, if a larger BW is need one can increase the cavity coupling by tuning the input impedance of the external circuit. The critical point, where maximum power is transferred between the cavity and the external circuit, occurs when the input impedance of the cavity matches the input impedance of the external transmission line.  For the series RLC circuit on resonance, this matching condition corresponds to 
\begin{equation}
    Z_0=Z_\mathrm{in}=R,
\end{equation}
where $Z_0$ is the impedance of the transmission line. The loaded Q at this critical point is, therefore,
\begin{equation}
    Q_L=\frac{1}{2\omega_0Z_0C}=\frac{Q_0}{2}.
\end{equation}
One can described the degree of coupling between the cavity and an external circuit by defining a coupling factor, $g$, such that,
\begin{equation}
    g=\frac{Q_0}{Q_L}-1.
\end{equation} 
When $g=1$ then $Q_L=Q_0/2$, and the cavity is said to be critically coupled as we described. If $Q_L<Q_0/2$, then the cavity is undercoupled to the transmission line, corresponding to $g<1$. Alternatively, if $Q_L>Q_0/2$, then $g>1$, and the cavity is overcoupled to the transmission line. Various specialized circuits can be used to tune the input impedance of the external circuit as seen by the cavity to achieve a wide range of different coupling factors based on the desired application of the cavity.

\section{The Cavity Approach to CRES}
\label{sec:chap6-cavity-approach}

\subsection{A Sketch of a Molecular Tritium Cavity CRES Experiment}

Resonant cavities can be used to perform CRES measurements, and they represent the current preferred technology by the Project 8 collaboration for the ultimate goal of a 40~meV neutrino mass measurement using CRES. The basic approach to a neutrino mass measurement using a resonant cavity and molecular tritium beta-decay source is illustrated by Figure \ref{fig:chap6-cav-cartoon}.
\begin{figure}[htbp]
    \centering
    \includegraphics*[width=0.7\textwidth]{figs/Chapter-6/230606_cavity_cartoon.png}
    \caption{\label{fig:chap6-cav-cartoon} Caption}
\end{figure}

At the core of the experiment is a large resonant cavity filled with tritium gas. In principle a sealed metallic cavity serves a dual purpose as the tritium containment vessel although the risks of exposure to radioactive tritium may necessitate a secondary dielectric containment vessel (such as fused scilica) inside the cavity volume. The filled cavity is then placed in a uniform magnetic field provided by a primary magnet that sets the values of the cyclotron frequencies for electrons emitted with energies near the tritium spectrum endpoint. When a beta-decay electron is produced in the cavity it is trapped using an additional set of magnetic coils that prevents the electrons from running into the cavity walls.

Electrons trapped inside the cavity volume do not radiate in the same way as electron in free-space. Effectively, the same boundary conditions that were used to derive the resonant modes of a cylindrical cavity in Section \ref{sec:chap6-resonant-cavities} apply to the radiation of the electron as well. If an electron is emitted with a kinetic energy that corresponds to a cyclotron frequency that matches a resonant frequency of the cavity, then the power radiated by the electron excites the corresponding resonance in the cavity. The strength of the electron's coupling to the cavity is given by the dot product between the electrons trajectory and the electric field vector of the resonant mode. However, if an electron is moving with a cyclotron frequency that is far from any resonant modes in the cavity, then radiation from the electron is suppressed. One can interpret this somewhat surprising effect as the metallic walls of the cavity reflecting the radiated energy back into the electron. 

To detect the electron the cavity is coupled to an external transmission line that leads to an amplifier and RF receiver chain similar to an antenna array based experiment. The coupling of the cavity resonance to the amplifier occurs through a coupling probe designed to resonate with the same mode or modes excited by the electron. In other resonant cavity systems this is often accomplished using a simple wire antenna,  which could be connected to the amplifier through a segment of coaxial cable. Alternatively, cavities are oftentimes coupled to waveguides or smaller external cavities using small holes or apertures cut into the main cavity wall. For CRES measurements, the placement of a wire coupling probe inside the cavity volume leads to additional scattering of electrons and eventually tritium atoms, therefore, the apertures are the preferred coupling method for cavity CRES experiments.

One of the attractive features of the CRES technique for neutrino mass measurement is the gain in statistics that comes from the differential nature of the tritium spectrum measurement. Initially, this seems incompatible with cavities, due to the narrow resonances of cavity modes giving relatively small bandwidth. However, by intentionally overcoupling to a single cavity mode one can achieve bandwidths of a few 10's of MHz, which is sufficient for a measurement of the tritium spectrum endpoint region.  

\subsection{Magnetic Field, Cavity Geometry, and Resonant Modes}

\subsubsection*{Magnetic Field and Volume Scaling}

For a CRES experiment, cylindrical cavities are a natural choice since they match the geometry of standard solenoid magnets, which are needed in order to produce the background magnetic field for CRES measurements. Furthermore, the cylindrical shape is compatible with a Halbach array, which is the leading choice of atom trapping technology for future atomic tritium experiments by the Project 8 collaboration. Cylindrical cavities also benefit from well-established machining practices that are able to achieve high geometric precision at large lengths scales. Currently, a cylindrical cavity is the preferred cavity shape for CRES measurements by Project 8, although, there are on-going efforts to investigate more complicated cavity designs that may offer advantages over the more standard geometry.

As we saw in Section \ref{sec:chap6-resonant-cavities}, the physical dimensions of the cavity are directly coupled to the resonant frequencies of the cavity. This dependency links the size of the cavity to the magnitude of the background magnetic field, because the magnetic field determines the cyclotron frequencies of trapped electrons. Specifically, as the size of the cavity is increased to accommodate larger volumes of tritium gas, the wavelengths and frequencies of the resonant modes increase and decrease respectively. This requires that the magnetic field also decrease in order to maintain coupling between electrons and the desired cavity mode. 

The required cavity size is ultimately determined by the required statistics in the tritium spectrum endpoint region. Because the gas density must be kept below a certain level to ensure that electrons have sufficient time to radiate before scattering, larger volumes become the only way to achieve higher event statistics. To achieve the sensitivity goals of Phase III and IV cavity volumes on the order of several cubic-meters are required, which pushes one towards frequencies in the range of 100's of MHz.

\subsubsection*{Single-mode Cavity CRES}

It is tempting to consider maintaining a high magnetic field while increasing the size of the cavity in order to increase the radiated power from trapped electrons for a potentially better SNR. However, if one were to maintain the same magnetic field while increasing the size of the cavity, the electrons would begin to couple to higher order modes with more complicated transverse geometries. The danger with this approach is that a complicated mode structure could introduce systematic errors into the CRES signals, for example by unpredicted mode hybridization or changes in the mode shapes from imperfections in the cavity construction, that would prevent reconstruction of the electron's starting kinetic energies with adequate resolution. For this reason, it may be ideal to operate with magnetic fields that give cyclotron frequencies near the fundamental frequency of the cavity, where the mode structure is relatively simple. In this frequency region it may be possible to perform CRES by coupling to only a single resonant mode, however, it is currently an open question if a single mode measurement will provide enough information about an individual electron to reconstruct the full event. Regardless, developing a solid understanding of the CRES phenomenology when an electron is coupling to a single mode will be a necessary step towards a future multi-mode cavity experiment.

\subsubsection*{Considerations for Resonant Mode Selection}

The design of a single-mode cavity experiment begs the question of which resonant mode is best for CRES measurements. There is an immediate bias towards low order $\mathrm{TE}_{nm}$ and $\mathrm{TM}_{nm}$ modes due to the multi-mode considerations discussed above. Additionally, there is a preference towards modes with longitudinal index $\ell=1$ with a single antinode along the vertical axis of the cylindrical cavity. The reason for this is that there is a phase change in the electric fields between antinodes that could lead to interference effects that destroy signal information when an electron is moving between antinodes. 

A second consideration for mode selection is the volumetric efficiency of the mode. Volumetric efficiency can be thought of as an integral over the volume of the cavity weighted by the relative amplitude of the mode. From the perspective of simply maximizing the volume useable for CRES measurements this integral would be as close to unity as possible. However, there is a requirement to reconstruct the position of the electrons inside the cavity volume so that the local magnetic fields can be used to convert the measured cyclotron frequency to a kinetic energy. With a single mode this necessarily requires a variable transverse mode amplitude, which lowers the volumetric efficiency, so that position of the electron in the cavity can be estimated from the average amplitude of the CRES signal. Longitudinal indices of $\ell=1$ have an advantage in volumetric efficiency over higher order $\ell$ modes, since there are only two longitudinal nodes, one at each end of the cavity. Therefore, the average coupling strength of trapped electrons as they oscillate axially is higher for $\ell=1$ modes.

The longitudinal variation in the mode strength is ultimately critical for achieving the energy resolution required for neutrino mass measurements. Correcting for the change in the average magnetic fields experienced by electrons with different pitch angles requires that information on the axial motion of the electron be encoded into the CRES signal. The longitudinal variation in the mode amplitude leads to amplitude modulation of the CRES signal with a frequency proportional to the electron's pitch angle. 

\begin{figure}
    \centering
    \begin{subfigure}{0.67\textwidth}
        \centering
        \includegraphics*[width=1\textwidth]{figs/Chapter-6/230610_resmodes_1ghz_ld4pt5_wide.png}
        \caption{}
    \end{subfigure}
    \hfill
    \begin{subfigure}{0.67\textwidth}
        \centering
        \includegraphics*[width=1\textwidth]{figs/Chapter-6/230610_resmodes_1ghz_ld4pt5.png}
        \caption{}
    \end{subfigure}
    \caption{}
\end{figure}

An additional factor for mode selection is the intrinsic or unloaded Q of the mode. In terms of SNR it is advantageous to use a mode with a very high $Q_0$, which is then highly overcoupled to achieve the necessary bandwidth to cover the tritium endpoint spectrum. This scheme leads to a decoupling of the physical cavity temperature from the effective noise temperature after the amplifier, which allows us to achieve adequate SNR without the requirement of cooling the entire cavity to single Kelvin temperatures. 

An example of a resonant mode that exhibits these traits is the $\mathrm{TE}_{011}$ mode. At present the $\mathrm{TE}_{011}$ mode is the preferred resonance for a single-mode cavity CRES experiment by the Project 8 collaboration. $\mathrm{TE}_{011}$ is a low order mode located in a region relatively far from other cavity modes. Furthermore, the separation of the $\mathrm{TE}_011$ mode can be improved by various mode-filtering techniques discussed in Section \ref{sec:chap6-mode-filtering-techniques} below. $\mathrm{TE}_{011}$ consists of a single longitudinal antinode that can provide pitch angle information in the form of amplitude modulation, and has an electric field with a radial profile given by the $J_0^\prime$ Bessel function allowing for radial position estimation. Lastly, the $\mathrm{TE}_{011}$ mode has a relatively high intrinsic Q compared to nearby modes, which helps with SNR. Unloaded Q's greater than 80000 are achievable for a 1~GHz $\mathrm{TE}_{011}$ resonance using a copper walled cavity. 

\subsection{Trade-offs Between the Antenna and Cavity Approaches}

The choice between cavities and antennas for large-scale CRES measurements is not without trade-offs. While both the antenna array and cavity approach are in their technical infancy, at present there are no known obstacles that would prevent either approach from being used for a large scale neutrino mass measurement. The emergent preference for cavities is partly driven by important practical considerations that could make a cavity based experiment significantly cheaper than an antenna experiment of similar size and scope. However, the switch to cavities also introduces new challenges less relevant to the antenna array, which must be solved in order for Project 8 to achieve its neutrino mass measurement goals. 

One of the major drawbacks of the antenna array approach compared with the cavity is the size and complexity of the data-acquisition system. A large-scale antenna array experiment would require $O(100)$ antennas all independently digitized at rates of $O(10)$ to $O(100)$~MHz. Since there is insufficient information in a single antenna channel to detect or reconstruct the CRES signal, the entire array output must be processed during the signal reconstruction. Because data storage becomes an issue with these data volumes, there is a requirement for some form of real-time signal reconstruction capable of detecting CRES signals buried in the thermal noise. As we discuss in Section \ref{sec:chap4-trigger-paper}, the computational cost of these real-time detection algorithms are potentially quite large for even a small scale antenna array experiment. However, the operating principle of a cavity experiment allows the CRES signal to be detected using only a single read-out channel digitized at rates of O(10)~MHz, which reduces the cost of the data acquisition system by many orders of magnitude.

From an engineering perspective, the simple geometry and thin-walls of a cylindrical cavity are much simpler to interface with the cryogenic and magnetic subsystems required for a CRES experiment. Whereas, the antenna array requires careful design and engineering to accommodate the antenna array and receiver electronics in proximity to the electron and eventually atom trapping magnets. Additionally, due to near-field interference effects the antenna array is unable to reconstruct CRES events within the reactive near-field distance of the antennas. Because atom trapping requirements require magnetic fields which correspond to cyclotron frequencies for endpoint electrons less than 1~GHz, the required stand-off distance leads to a significant loss in useable experiment volume, necessitating larger and more expensive magnets.

Another advantage to the cavity approach is the relatively compact sideband structure, which is a result of the low modulation index for cavity CRES signals. The axial motion in an antenna array experiment leads to frequency modulation and sidebands. The shape of the sideband structure is determined by the modulation index, $h=\frac{\Delta f}{f_a}$, where $\Delta f$ is the size of the frequency deviation and $f_a$ is the axial frequency. The large electron traps required for a cubic-meter-scale experiment leads to high modulation indices, which causes the signal spectrum to be made up of numerous low power sidebands that make reconstruction and detection challenging. This behavior was observed in simulations of the FSCD in which carrier power decreased with pitch angle due to the increase in modulation index (see Figure \ref{fig:signal_post_bf_example}). For cavities, however, the modulation index remains near $h=1$ even for very long magnetic traps due to the high phase velocity in cavities relative to the axial velocity of the electron. This results in an almost ideal spectrum shape that has a strong carrier frequency with a few sidebands whose relative amplitudes encode pitch angle information. 

A potential downside of the cavity approach is the apparent difficulty of estimating the position of the electron using only the coupling of the electron to a single mode. The amplitude of the $\mathrm{TE}_{011}$ mode is completely independent of the azimuthal coordinate, therefore, position reconstruction using the $\mathrm{TE}_{011}$ mode is only able to estimate the radial position of the electron. This position degeneracy may lead to magnetic field uniformity requirements that are extremely challenging to meet due to mechanical uncertainties in cavity and magnet construction, as well as uncertainties caused by nuisance external magnetic fields such as the Earth's field and magnetic fields from building materials. A multi-mode cavity experiment may provide a way to extract more precise information on the position of the electron by analyzing the coupling of the electron to several modes that overlap in different ways.

\section{Single-mode Resonant Cavity Design and Simulations}
\label{sec:chap6-single-mode-cavity-sims}

The single-mode cylindrical cavities envisioned for the Phase III and IV experiments must be carefully engineered in order to measure the neutrino mass with the desired sensitivity. In this section I summarize some simulation studies performed to analyze early design concepts for a single-mode cavity. The primary tool for these investigations was Ansys HFSS, which was also used for the development of the SYNCA antenna described in Section \ref{sec:SYNCA}. 

\subsection{Open Cylindrical Cavities with Coaxial Terminations}

\begin{figure}[htbp]
    \centering
    \includegraphics*[width=0.6\textwidth]{figs/Chapter-6/230606_open_cavity_image.png}
    \caption{Caption}
\end{figure}

\begin{figure}[htbp]
    \centering
    \includegraphics*[width=0.7\textwidth]{figs/Chapter-6/230606_open_cavity_sketch.png}
    \caption{Caption}
\end{figure}

\subsection{Mode Filtering Techniques}
\label{sec:chap6-mode-filtering-techniques}

\begin{figure}[htbp]
    \centering
    \begin{subfigure}{0.7\textwidth}
        \centering
        \includegraphics*[width=\textwidth]{figs/Chapter-6/230608_insulator_mode_filter_cartoon.png}
        \caption{}
    \end{subfigure}
    \hfill
    \begin{subfigure}{0.7\textwidth}
        \centering
        \includegraphics*[width=\textwidth]{figs/Chapter-6/230608_grooved_mode_filter_cartoon.png}
        \caption{}
    \end{subfigure}
    \caption{}
\end{figure}

\subsection{Simulations of Open, Mode-filtered Cavities}

\begin{figure}[htbp]
    \centering
    \includegraphics*[width=0.7\textwidth]{figs/Chapter-6/230610_cavity_variations.png}
    \caption{}
\end{figure}

\begin{figure}[htbp]
    \centering
    \includegraphics*[width=0.9\textwidth]{figs/Chapter-6/230610_cavity_variation_eigenmodes_linear_annotate.png}
    \caption{}
\end{figure}

\section{Single-mode Resonant Cavity Measurements}
\label{sec:chap6-single-mode-cavity-measurement}

\subsection{Cavities and Setup}

\begin{figure}[htbp]
    \centering
    \includegraphics*[width=0.7\textwidth]{figs/Chapter-6/230612_toy_cavity_cartoon.png}
    \caption{}
\end{figure}


\begin{figure}[htbp]
    \centering
    \begin{subfigure}{0.48\textwidth}
        \includegraphics*[width=\textwidth]{figs/Chapter-6/230612_open_cav_meas_image.jpg}
        \caption{}        
    \end{subfigure}
    \hfill
    \begin{subfigure}{0.48\textwidth}
        \includegraphics*[width=\textwidth]{figs/Chapter-6/230612_filtered_cav_meas_image.jpg}
        \caption{}
    \end{subfigure}
    \hfill
    \begin{subfigure}{0.48\textwidth}
        \includegraphics*[width=\textwidth]{figs/Chapter-6/230612_coupling_loop.png}
        \caption{}
    \end{subfigure}
    \caption{}
\end{figure}

\subsection{Results and Discussion}

\begin{figure}[htbp]
    \centering
    \includegraphics*[width=0.7\textwidth]{figs/Chapter-6/230612_simulated_toy_cav_no_term_modes_annotated.png}
    \caption{}
\end{figure}

\begin{figure}[htbp]
    \centering
    \begin{subfigure}{0.67\textwidth}
        \includegraphics*[width=\textwidth]{figs/Chapter-6/210922_open_cavity_no_term_TM_vs_TE.png}
        \caption{}
    \end{subfigure}
    \hfill
    \begin{subfigure}{0.67\textwidth}
        \includegraphics*[width=\textwidth]{figs/Chapter-6/210922_open_cavity_no_term_TM_vs_TE_zoom.png}
        \caption{}
    \end{subfigure}
    \caption{}
\end{figure}

\begin{figure}[htbp]
    \centering
    \begin{subfigure}{0.67\textwidth}
        \includegraphics*[width=\textwidth]{figs/Chapter-6/210922_open_cavity_with_and_without_term_TE.png}
        \caption{}
    \end{subfigure}
    \hfill
    \begin{subfigure}{0.67\textwidth}
        \includegraphics*[width=\textwidth]{figs/Chapter-6/210922_open_cavity_with_and_without_term_TM.png}
        \caption{}
    \end{subfigure}
    \caption{}
\end{figure}

\begin{figure}[htbp]
    \centering
    \begin{subfigure}{0.67\textwidth}
        \includegraphics*[width=\textwidth]{figs/Chapter-6/220601_TE_resonance_filtering.png}
        \caption{}
    \end{subfigure}
    \hfill
    \begin{subfigure}{0.67\textwidth}
        \includegraphics*[width=\textwidth]{figs/Chapter-6/220505_TM_resonance_filtering.png}
        \caption{}
    \end{subfigure}
    \caption{}    
\end{figure}
