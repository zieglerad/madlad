% !TEX root = ../YourName-Dissertation.tex

\chapter{Introduction} 
\label{chap:introduction}

\section{Summary}

Neutrinos are one of the fundamental particles that comprise the standard model of particle physics and account for a significant fraction of the matter in the universe. Neutrinos are the most abundant fermions in the universe, but due to their weak interactions neutrinos seldom interact with other particles. Regardless, neutrinos play a unique role in the evolution of the early-universe, therefore, a detailed understanding of the properties of the neutrino is important to understanding the cosmology of the universe as well as understanding the universe at the fundamental particle physics scale.

Unlike other fermions it was unclear that neutrinos had nonzero mass until neutrino flavor oscillations were definitively observed in the late 90's and early 00's. Flavor oscillations require that neutrinos experience time so that when acted upon by the time-evolution operator the initial neutrino state can evolve to a new flavor state. This implies that the neutrino flavor states are really a superposition of at least three separate neutrino states with well-defined masses.Measurements of neutrino oscillations that have taken place over the past couple of decades have measured the differences between neutrino mass eigenstates with increasing precision. However, oscillation measurements cannot tell us the mass scale of the neutrinos, which is required in order to measure the absolute neutrino masses.

The neutrino mass scale remains an unknown quantity in the standard model of particle physics. The value of the neutrino mass influences the evolution of the early universe and is likely relevant to the energy-scale of new physics responsible for the factor of $10^{-6}$ difference between the neutrino and electron masses. A model-independent way to measure the neutrino mass is to measure the tritium beta-decay spectrum near its endpoint. Energy conservation requires that the neutrino mass carry away some kinetic energy from the beta-decay electron in the form of its mass, which causes a distortion in the shape of the tritium beta-decay spectrum near the endpoint. The isotope tritium has many advantages for this measurement, and has been used by the KATRIN collaboration to perform the most sensitive direct neutrino mass measurement to date.

KATRIN represents the state-of-the-art experiment in the current generation of neutrino mass direct measurement experiments and has a final projected sensitivity to neutrino masses $m_\nu < 200$~meV. This sensitivity does not fully exhaust the allowed parameter space of neutrino masses under the normal and inverted neutrino mass ordering scenarios, which motivates the development of a next generation of neutrino mass measurement experiments. 

The Project 8 collaboration is developing a next-generation neutrino mass direct measurement experiment designed to be sensitive to $m_\nu<40$~meV. This sensitivity is sufficient to exhaust the range of neutrino masses allowed under the inverted mass ordering regime. Project 8 intends to achieve its sensitivity goal utilizing two technologies that are novel to the space of direct neutrino mass measurement --- atomic tritium and cyclotron radiation emission spectroscopy (CRES). Atomic tritium is required in order to avoid systematic broadening the tritium beta-decay spectrum caused by the final state of the $^3\textrm{He}^+$-T molecule, and the CRES technique enables a differential measurement of the tritium spectrum that is background-free and able to be directly integrated with the atomic tritium source.

The Project 8 collaboration is currently engaged in a research and development program intended to simultaneously develop the atomic tritium and CRES technologies so that they can be combined in a next-generation experiment. This past year (2022) Project 8 has used the CRES technique to measure the molecular tritium beta-decay spectrum and place an upper limit on the neutrino mass: $m_\beta\leq152$~eV. This measurement, while not competitive scientifically, represents the first proof-of-principle that the CRES technique can be used to measure the neutrino mass.

The future goals of the Project 8 collaboration are to develop the technologies and techniques necessary to scale-up the volumes in which CRES measurements can be performed. Project 8's first neutrino mass measurement with CRES utilized a measurement volume on the cubic-centimeter scale, however, sensitivity calculations estimate that an experiment sensitive to neutrino masses of 40~meV will require several tens of cubic-meters of experiment volume filled with atomic tritium. Developing a new approach to performing CRES measurements that can be successfully scaled to these volumes is a necessary step towards Project 8's neutrino mass measurement goal, and is the primary topic of my dissertation research.

A parallel development is the technology necessary to produce, cool, trap, and recirculate a supply of atomic tritium that is compatible with CRES measurements. The atomic tritium system is equally important as the large-volume CRES measurement technology, but it will not be the focus of this dissertation since I did not contribute significantly to this effort.

The Project 8 collaboration has identified two scalable approaches to neutrino mass measurement using the CRES technique. One approach is to use an array of antennas that surrounds a volume of trapped atomic tritium that can perform CRES measurements by collection the cyclotron radiation emitted by beta-decay electrons into free-space. The other approach uses a resonant cavity filled with atomic tritium to perform CRES by measuring the excitation of resonant cavity modes caused by the motion of electrons trapped inside the cavity volume. 

The cavity and antenna approaches to CRES have been studied in detail over the past five years, and, while both approaches offer a physically viable path towards a 40~meV neutrino mass measurement the collaboration has elected to pursue the cavity approach for the foreseeable future. The major advantage of the cavity approach is a significant reduction in the cost and complexity of the experiment design and data analysis, which provides a less risky path towards Project 8's scientific goals. 

In this dissertation I summarize my most impactful contributions to the research and development of antenna array and cavity CRES. In short these contributions are
\begin{itemize}
    \item the development and analysis of signal reconstruction algorithms for antenna array CRES, which provided key inputs to sensitivity analyses of antenna array CRES experiments,
    \item the development of a specialized antenna designed to synthesize fake CRES radiation, which enabled bench-top testing and validation of the antenna array CRES technique,
    \item the development of an open-cavity design for CRES measurement whose mode structure can be tuned using perturbations that modify the impedance of the cavity walls. The development of this cavity concept was one of many developments that eventually lead to the adoption of cavities as the CRES technology of choice for the future of Project 8. 
\end{itemize}

\section{Outline}

The outline of this dissertation is as follows. In Chapter 2 I provide an introduction to the basic physics of neutrinos and beta-decay, which provides context for a discussion of various methods to measure the neutrino absolute mass scale. 

Chapter 3 is an overview of the CRES technique and the Project 8 collaboration. I highlight the Project 8 Phase II experiment, which was the first measurement of the tritium beta-decay spectrum with CRES, and I discuss the planned research and development for an antenna array CRES experiment in Phase III of the Project 8 collaboration's experiment plan. I end Chapter 3 with a discussion of the pilot-scale and Phase IV experiments, that will combine a scalable CRES measurement technology with atomic tritium and measure the neutrino mass with 40~meV sensitivity.

Chapter 4 discusses the first of the contributions mentioned above, which is the development of signal reconstruction techniques for antenna array CRES and an antenna array demonstrator experiment called the FSCD. I discuss the important tools that Project 8 uses to simulate antenna array CRES before introducing three signal reconstruction algorithms that can be used to detect CRES signals using the array. I end Chapter 4 with a paper that summarizes a detailed analysis and comparison of the signal detection performance of each algorithm.

Chapter 5 describes my contributions to the development of antennas and an antenna measurement system for Project 8, which is the second major contribution of this dissertation. I begin with a general overview of basic principle of antennas and antenna measurements, before including a paper that describes the development of unique antenna designed to mimic the cyclotron radiation emitted by electrons in free-space when trapped in a magnetic field. I call this antenna the synthetic cyclotron radiation antenna (SYNCA) and its main purpose is to serve a fake electron for laboratory validation measurements of Project 8's antenna array CRES simulations. Chapter 5 ends with an overview of laboratory measurements of a prototype antenna array that were compared with simulations to provide upper bounds on reconstruction errors caused by imperfections in real-life measurements.

Chapter 6 discusses the cavity approach to CRES, which was adopted as the preferred CRES technology for Phase IV late into my dissertation work. The chapter stars by discussing resonant cavities in general before introducing the operating principles of the cavity approach to CRES. I end the chapter by discussing a study of and open-cavity design that could be used for CRES measurements and integrated with atomic tritium and an electron gun calibration source for the pilot-scale and Phase IV experiments.

Finally, in Chapter 7 I conclude by briefly discussing the future directions of the Project 8 collaboration as we continue towards a direct measurement of the neutrino mass. 




