% !TEX root = ../YourName-Dissertation.tex

\chapter{Introduction and Summary} 
\label{chap:introduction}

Neutrinos are one of the fundamental particles that comprise the standard model of particle physics and account for a significant fraction of the matter in the universe. Neutrinos are the most abundant fermions in the universe, but due to their weak interactions neutrinos seldom interact with other particles. Regardless, neutrinos play a unique role in the evolution of the early-universe, therefore, a detailed understanding of the properties of the neutrino is important to understanding the cosmology of the universe as well as understanding the universe at the fundamental particle physics scale.

Unlike other fermions it was unclear that neutrinos had nonzero mass until neutrino flavor oscillations were definitively observed in the late 90's and early 00's. Flavor oscillations require that neutrinos experience time so that when acted upon by the time-evolution operator the initial neutrino state can evolve to a new flavor state. This implies that the neutrino flavor states are really a superposition of at least three separate neutrino states with well-defined masses.Measurements of neutrino oscillations that have taken place over the past couple of decades have measured the differences between neutrino mass eigenstates with increasing precision. However, oscillation measurements cannot tell us the mass scale of the neutrinos, which is required in order to measure the absolute neutrino masses.

The neutrino mass scale remains an unknown quantity in the standard model of particle physics. The value of the neutrino mass influences the evolution of the early universe and is likely relevant to the energy-scale of new physics responsible for the factor of $10^{-6}$ difference between the neutrino and electron masses. A model-independent way to measure the neutrino mass is to measure the tritium beta-decay spectrum near its endpoint. Energy conservation requires that the neutrino mass carry away some kinetic energy from the beta-decay electron in the form of its mass, which causes a distortion in the shape of the tritium beta-decay spectrum near the endpoint. The isotope tritium has many advantages for this measurement, and has been used by the KATRIN collaboration to perform the most sensitive direct neutrino mass measurement to date.

KATRIN represents the state-of-the-art experiment in the current generation of neutrino mass direct measurement experiments and has a final projected sensitivity to neutrino masses $m_\nu < 200$~meV. This sensitivity does not fully exhaust the allowed parameter space of neutrino masses under the normal and inverted neutrino mass ordering scenarios, which motivates the development of a next generation of neutrino mass measurement experiments. 

The Project 8 collaboration is developing a next-generation neutrino mass direct measurement experiment designed to be sensitive to $m_\nu<40$~meV. This sensitivity is sufficient to exhaust the range of neutrino masses allowed under the inverted mass ordering regime. Project 8 intends to achieve its sensitivity goal utilizing two technologies that are novel to the space of direct neutrino mass measurement --- atomic tritium and cyclotron radiation emission spectroscopy (CRES). Atomic tritium is required in order to avoid systematic broadening the tritium beta-decay spectrum caused by the final state of the $^3\textrm{He}^+$-T molecule, and the CRES technique enables a differential measurement of the tritium spectrum that is background-free and able to be directly integrated with the atomic tritium source.

The Project 8 collaboration is currently engaged in a research and development program intended to simultaneously develop the atomic tritium and CRES technologies so that they can be combined in a next-generation experiment. This past year (2022) Project 8 has used the CRES technique to measure the molecular tritium beta-decay spectrum and place an upper limit on the neutrino mass: $m_\beta\leq152$~eV. This measurement, while not competitive scientifically, represents the first proof-of-principle that the CRES technique can be used to measure the neutrino mass.

The future goals of the Project 8 collaboration are to develop the technologies and techniques necessary to scale-up the volumes in which CRES measurements can be performed. Project 8's first neutrino mass measurement with CRES utilized a measurement volume on the cubic-centimeter scale, however, sensitivity calculations estimate that an experiment sensitive to neutrino masses of 40~meV will require several tens of cubic-meters of experiment volume filled with atomic tritium. Developing a new approach to performing CRES measurements that can be successfully scaled to these volumes is a necessary step towards Project 8's neutrino mass measurement goal, and is the primary topic of my dissertation research.

A parallel development is the technology necessary to produce, cool, trap, and recirculate a supply of atomic tritium that is compatible with CRES measurements. The atomic tritium system is equally important as the large-volume CRES measurement technology, but it will not be the focus of this dissertation since I did not contribute significantly to this effort.

The Project 8 collaboration has identified two scalable approaches to neutrino mass measurement using the CRES technique. One approach is to use an array of antennas that surrounds a volume of trapped atomic tritium that can perform CRES measurements by collection the cyclotron radiation emitted by beta-decay electrons into free-space. The other approach uses a resonant cavity filled with atomic tritium to perform CRES by measuring the  

In Chapter 2\ldots

In Chapter 3\ldots

In Chapter 4\ldots

In Chapter 5\ldots

In Chapter 6\ldots

In Chapter 7\ldots




