% !TEX root = ../YourName-Dissertation.tex

\chapter{Introduction} 
\label{chap:introduction}

\section{Summary}

Neutrinos are one of the fundamental particles in the standard model of particle physics and account for a significant fraction of the matter in the universe. Neutrinos are the most abundant fermions in the universe, but due to their weak interactions neutrinos seldom interact with other particles. Regardless, neutrinos play a unique role in the evolution of the early-universe, and a detailed understanding of the properties of the neutrino is key to understanding the universe at cosmological scales, as well as how the universe operates on smallest particle physics scale.

It was uncertain that neutrinos had nonzero mass until vacuum neutrino flavor oscillations were observed in the late 90's and early 00's. A simple relativistic argument as to why oscillations are evidence for neutrino masses is that oscillations imply neutrinos experience time, which means that they do not propagate at the speed of light, therefore the masses of the neutrinos must be non-zero. Current neutrino oscillation data supports that neutrino flavor states are actually a superposition of three separate neutrino states with well-defined masses. Measurements of neutrino oscillations that have taken place over the past couple of decades have measured the differences between neutrino mass eigenstates with increasing precision. However, oscillation measurements cannot tell us the mass scale of the neutrinos, which is required in order to measure the absolute neutrino masses.

The neutrino mass scale remains an unknown quantity in the standard model of particle physics. The value of the neutrino mass influences the evolution of the early universe and is likely relevant to the energy-scale of new physics responsible for the factor of $10^{-6}$ difference between the neutrino and electron masses. A model-independent way to measure the neutrino mass is to measure the tritium beta-decay spectrum near its endpoint. Energy conservation requires that the neutrino carry away some kinetic energy from the decay in the form of its mass, which causes a distortion in the shape of the tritium beta-decay kinetic energy spectrum near the endpoint. The isotope tritium has many advantages for this measurement, and has been used by the KATRIN collaboration to perform the most sensitive direct neutrino mass measurement to date.

KATRIN represents the state-of-the-art in the current generation of neutrino mass direct measurement experiments with a projected neutrino mass sensitivity  of $m_\nu < 200$~meV. This sensitivity does not fully exhaust the allowed parameter space of neutrino masses under the normal and inverted neutrino mass ordering scenarios, which motivates the development of a next generation of neutrino mass measurement experiments. 

The Project 8 collaboration is developing a next-generation neutrino mass experiment with a goal neutrino mass sensitivity of $m_\nu<40$~meV. This sensitivity is sufficient to exhaust the range of neutrino masses allowed under the inverted mass ordering regime. Project 8 intends to achieve its sensitivity goal by utilizing two technologies that are novel to the space of direct neutrino mass measurements --- atomic tritium and cyclotron radiation emission spectroscopy (CRES). Atomic tritium is required in order to avoid systematic broadening the tritium beta-decay spectrum caused by the final state of the $^3\textrm{He}^+$-T molecule. The CRES technique enables a high-resolution, differential measurement of the tritium spectrum that is background-free and able to be directly integrated with the atomic tritium source.

The Project 8 collaboration is currently engaged in a research and development program intended to simultaneously develop the atomic tritium and CRES technologies so that they can be combined in a next-generation experiment. This past year (2022) Project 8 has used the CRES technique to measure the molecular tritium beta-decay spectrum and place an upper limit on the neutrino mass: $m_\beta\leq152$~eV. This measurement, while not competitive scientifically, represents the first proof-of-principle demonstration that the CRES technique can be used to measure the neutrino mass.

The future goals of the Project 8 collaboration are to develop the technologies and techniques necessary to scale-up the volume in which CRES measurements can be performed. Project 8's first neutrino mass measurement with CRES utilized a measurement volume on the cubic-centimeter scale, however, sensitivity calculations estimate that an experiment sensitive to neutrino masses of 40~meV will require $\sim100$~m$^3$ of experiment volume filled with atomic tritium. Developing a new approach to performing CRES measurements that can be successfully scaled to these volumes is a necessary step towards Project 8's neutrino mass measurement goal, and is the primary goal my dissertation research is directed at.

A parallel development is the technology necessary to produce, cool, trap, and recirculate a supply of atomic tritium that is compatible with CRES measurements. The atomic tritium system is equally important as the large-volume CRES measurement technology, but will not be discussed at depth here.

The Project 8 collaboration has identified two scalable approaches to neutrino mass measurement using the CRES technique. One approach is to use an array of antennas that surrounds a volume of trapped atomic tritium that can perform CRES measurements by collection the cyclotron radiation emitted by beta-decay electrons into free-space. The other approach uses a resonant cavity filled with atomic tritium to perform CRES by measuring the excitation of resonant cavity modes caused by the motion of electrons trapped inside the cavity volume. 

The cavity and antenna approaches to CRES have been studied in detail over the past five years, and, while both approaches offer a physically viable path towards a 40~meV neutrino mass measurement, the collaboration has elected to pursue the cavity approach for the foreseeable future. The major advantage of the cavity approach is a significant reduction in the cost and complexity of the experiment design and data analysis, which provides a lower risk path to Project 8's scientific goals. 

In this dissertation I summarize my most impactful contributions to the research and development of antenna array and cavity CRES. In short these contributions are
\begin{itemize}
    \item the development and analysis of signal reconstruction algorithms for antenna array CRES, which provide key inputs to sensitivity analyses of antenna array CRES experiments.
    \item The development of a specialized antenna, designed to synthesize fake CRES radiation, which enables bench-top testing and validation of the antenna array CRES technique.
    \item The development of an open-cavity design for CRES measurement, whose mode structure can be tuned using perturbations that modify the impedance of the cavity walls. The development of this cavity concept was one of many developments that eventually lead to the adoption of cavities as the CRES technology of choice for the future of Project 8. 
\end{itemize}

\section{Outline}

The outline of this dissertation is as follows. In Chapter 2 I provide an introduction to the basic physics of neutrinos and beta-decay, which provides context for a discussion of various methods to measure the neutrino absolute mass scale. 

Chapter 3 is an overview of the CRES technique and the Project 8 collaboration. Project 8's experimental program is organized into four phases. The first phase completed in 2015 before I began my dissertation work, so I begin by highlighting the Project 8's first measurement of the tritium beta-decay spectrum with CRES. Next, I discuss the planned research and development for an antenna array CRES experiment for the upcoming phase of Project 8's experimental program. I end Chapter 3 with a discussion of Project 8's pilot-scale and final phase experiments, that will combine a scalable CRES measurement technology with atomic tritium and measure the neutrino mass with 40~meV sensitivity.

Chapter 4 discusses the first of my contributions mentioned above, which is the development of signal reconstruction techniques for antenna array CRES and an antenna array demonstrator experiment called the FSCD. I discuss the key tools that Project 8 uses to simulate antenna array CRES before introducing signal reconstruction algorithms that can be used to detect CRES signals using the array. I end Chapter 4 with a detailed analysis and comparison of the signal detection performance of each algorithm, as reported in a paper I have authored.

Chapter 5 describes my contributions to the development of antennas and an antenna measurement system for Project 8, which is the second major contribution of this dissertation. I begin with a general overview of basic principle of antennas and antenna measurements, and describe the development, as reported in another paper I have authored, of unique antenna designed to mimic the cyclotron radiation emitted by electrons in free-space. I call this antenna the synthetic cyclotron radiation antenna (SYNCA) and its main purpose is to serve as a fake electron for laboratory validation measurements of Project 8's antenna array CRES simulations. Chapter 5 ends with an overview of laboratory measurements of a prototype antenna array using the SYNCA, which were compared with simulations to provide upper bounds on reconstruction errors caused by imperfections in real-life measurements.

Chapter 6 discusses the cavity approach to CRES, which was adopted as the preferred CRES technology for Phase IV late into my dissertation work. The chapter starts by discussing resonant cavities in general before introducing the operating principles of the cavity approach to CRES. I end the chapter by discussing a study of an open-cavity design that could be used for CRES measurements and integrated with atomic tritium and an electron gun calibration source for the pilot-scale and Phase IV experiments.

Finally, in Chapter 7 I conclude by briefly discussing the future directions of Project 8 as development proceeds towards a direct measurement of the neutrino mass. 




