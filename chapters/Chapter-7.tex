% !TEX root = ../YourName-Dissertation.tex

\chapter{Conclusion and Future Prospects}

In this dissertation we have discussed research and development efforts towards the development of a scalable CRES measurement technology that can be used to build a CRES experiment at cubic-meter scales with sensitivity to neutrino masses of 40~meV. The primary contributions of my dissertation are the development and analysis of signal reconstruction algorithms for an antenna array based CRES experiment \cite{p8phase3trigger}, which leads to estimates of the neutrino mass sensitivity; the development of a synthetic cyclotron radiation antenna (SYNCA) \cite{p8synca}, which allowed for laboratory validation of antenna array CRES simulation models \cite{p8jugaad}; and the development of an open-ended cavity design compatible with atomic tritium for a cavity based CRES experiment. A measurable impact of this work is the transition of the Project 8 collaboration's experimental plan from an antenna array based approach to a cavity based approach, where my work played a key role in demonstrating the significantly higher cost and complexity of the antenna array experiment.

The transition from antenna arrays to cavities requires a new set of demonstrator experiments to make incremental progress towards a 40~meV measurement of the neutrino mass. At the time of writing, the near-term plan of Project 8 is to design and construct a small-scale cavity CRES experiment utilizing the 1~T magnet installed in the UW-Seattle. This cavity is designed to have a TE011 resonance with a frequency of about 26~GHz with a length-to-diameter ratio that mimics the larger cavities intended for the pilot-scale and Phase IV experiments. The goal of this experiment is to demonstrate cavity CRES as well as validate models of CRES systematics using electrons from $^{83m}$Kr and an electron gun. Though the primary goal is demonstration, near-term physics measurements are available in the form of high-resolution measurements of the $^{83m}$Kr conversion spectrum of interest to the KATRIN collaboration.

Furthermore, Project 8 is currently constructing a low-frequency CRES setup located at Yale University to better understand the principles of cavity based CRES at lower magnetic fields. The Low, UHF Cavity Krypton Experiment at Yale (LUCKEY) is a 1.5~GHz cavity CRES experiment the will use conversion electrons from $^{83m}$Kr to perform CRES measurements at the lowest frequencies ever attempted with the technique. LUCKEY will validate frequency scaling models developed by Project 8 and will pave the way for the future Low-Frequency Apparatus (LFA), which will be a larger, 1~GHz cavity CRES experiment that includes a molecular tritium source. The target for the LFA is a measurement of the neutrino mass with a sensitivity of approximately 0.2~eV, which will build towards the atomic pilot-scale CRES experiment.

In parallel to the development of cavity CRES is the development of the atomic tritium source. Recent demonstrations of the production of atomic hydrogen are excellent steps towards the atomic tritium production needed for the pilot-scale experiment. One area of future study includes the development of a more detailed unstanding of the efficiency of atomic hydrogen production. Near-term plans include the development of a magnetic, evaporatively cooled beamline, as well as the prototyping of a Halbach array atoms trap. Nearly all of the components of the atomic tritium system will require demonstration before the complete system can be built. The long-term goal of the atomic tritium work is to construct a full atomic tritium prototype that demonstrates the production, cooling, trapping, and recycling of tritium at the rates needed for the pilot-scale experiment.

More broadly, the long-term goal of the Project 8 collaboration is to fully develop both the atomic tritium and cavity CRES technologies so that both can be combined in a pilot-scale CRES experiment. It is envisioned that this process will take approximately 10~years for both atomic tritium and cavity CRES. After these developments comes the pilot-scale experiment which will be the first CRES experiment that simultaneously demonstrates all the required technologies for Phase IV. Scaling to Phase IV with cavity CRES will require the construction of multiple copies (approximately 10) of the pilot-scale experiment to obtain sufficient statistics for 40~meV sensitivity.

Development of the CRES experimental technique by Project 8 has led to new experiments utilizing the CRES technique for basic physics research, such as the $^6$He-CRES collaboration \cite{cresHe6}, and has also found applications as a new approach to x-ray spectroscopy \cite{cresXrayKazkaz}. Recently, a new experimental effort called CRESDA has begun in the UK to develop new quantum technologies applied to CRES measurements for the neutrino mass \cite{cresda}. This flourishing of new experimental efforts based on the CRES technique is likely to continue as Project 8 continues to develop the technique towards its neutrino mass measurement goal.

